\documentclass[a4paper,sffamily,12pt]{article}

\usepackage[T1]{fontenc}
\usepackage[french]{babel}
\usepackage[utf8]{inputenc}

% Customization des listes
\usepackage{enumitem}
\usepackage{pifont}

% Insertion d'image
\usepackage{graphicx}

% Création de lien
\usepackage[colorlinks,linkcolor=blue]{hyperref}

% Formatage des titres de sections
\usepackage{titlesec}
\titleformat{\section}
  {\normalfont\Large\bfseries\sffamily}{\thesection.}{0.33em}{}[\hrule]
\titleformat{\paragraph}
  {\normalfont\bfseries\sffamily}{\theparagraph.}{0.33em}{}
 
 % tableau rectangle 
%\usepackage{slashbox}
%\usepackage{tabularx}

 % En-tête
\usepackage{fancyhdr}
\pagestyle{fancy}
\renewcommand\headrulewidth{1pt}
\fancyhead[L]{Base de données évoluées}
\fancyhead[R]{$X1I1050$}

% Permet de mettre du texte au dessus du titre
\usepackage{titling}
\renewcommand{\maketitlehooka}{\noindent COUILLEROT Carol \hfill  \hfill \\ MAHIER Loïc \hfill \\ PHALAVANDISHVILI  Demetre}

% Titre
\title{\vspace{\fill}\LARGE\bfseries\sffamily Rapport de projet\protect\footnote{rapport réalisé sous \LaTeX} \vspace{\fill}}

\begin{document}

	\date{} % Supprime la date
	\maketitle % Affiche le titre

	\thispagestyle{fancy} % Permet de mettre le titre sur la page ''fancy''
	
	\newpage
			
	\renewcommand{\contentsname}{Sommaire}
	\tableofcontents
	
	\newpage
	
	\section{Introduction}

		\vspace{0.5cm}
		
		L'objectif de ce projet est de réaliser un entrepôt de données (OLAP) ainsi que des requêtes intéressantes sur un ou plusieurs jeu de données libres (open data). Pour ce faire nous avons choisis deux jeux de données : le premier sur les hébergements collectifs en France, c'est à dire les hôtels, les campings et les résidences avec des informations sur leur location, leur classement (étoile) et leur capacité d'accueil notamment. Le deuxième jeu de donnée recueil toutes les communes de France, en indiquant leur population, leur superficie, leur code postal ainsi que leur département et leur région entre autre. Ce dernier nous permet d'affiner nos requête, d'en proposer des plus complexes mais aussi de pouvoir faire des regroupements et des classements par région et par département. Nous avons choisis de réaliser ce projet en PL/SQL ainsi que d'utiliser Talend pour nettoyer nos données et concevoir nos tables relationnelles. \\


	\section{Choix des données}				

		\vspace{0.5cm}
		
			
		
		


		
	\section{Schéma en étoile}

		\vspace{0.5cm}
		
		
		
		
		
				
	\section{Intégration avec Talend}

		\vspace{0.5cm}
		
		
		
		
		
		
			
	\section{Requêtes}

		\vspace{0.5cm}
		
		
		
		
		
																
	\section{Conclusion}

		\vspace{0.5cm}
		
		
		
		
				

		\newpage
	
	\section{Annexe}
					
\end{document}
